\documentclass[14pt, a4paper]{extreport}

\newcommand{\comment}[1]{}

\usepackage{hyperref}
\usepackage{graphicx}
\usepackage[utf8]{inputenc}
\usepackage[T1, T2A]{fontenc}
\usepackage{fixltx2e}
\usepackage{grffile}
\usepackage{longtable}
\usepackage{wrapfig}
\usepackage{rotating}
\usepackage[normalem]{ulem}
\usepackage{amsmath}
\usepackage{textcomp}
\usepackage{amssymb}
\usepackage{capt-of}
\usepackage{listings}
\usepackage{totcount}
\usepackage[figure,table]{totalcount}
\usepackage{etoolbox}
\usepackage{tocvsec2}
\usepackage{booktabs}

\setcounter{secnumdepth}{2}

\regtotcounter{page}

\newtotcounter{citenum}
\def\oldcite{}
\let\oldcite=\bibcite
\def\bibcite{\stepcounter{citenum}\oldcite}

\newtotcounter{attachcnt}

\usepackage{indentfirst}
\usepackage[left=3cm,right=1cm,
  top=2cm,bottom=2cm,bindingoffset=0cm]{geometry}
\usepackage[nodisplayskipstretch]{setspace}
\onehalfspacing
\sloppy
%\parindent=1cm

\usepackage{enumitem}
\setlist{nolistsep}

\usepackage{titlesec}

\usepackage{etoolbox}

\makeatletter
\patchcmd{\ttlh@hang}{\parindent\z@}{\parindent\z@\leavevmode}{}{}
\patchcmd{\ttlh@hang}{\noindent}{}{}{}
\makeatother


\titleformat{\chapter}[display]{\filcenter\bfseries\Large}{}{8pt}{}{}
\titleformat{\section}{\bfseries\Large}{\thesection}{1em}{}{}
\titleformat{\subsection}{\bfseries\normalsize}{\thesubsection}{1em}{}{}
\titleformat{\subsubsection}{\bfseries\normalsize}{\thesubsubsection}{1em}{}{}

% Настройка вертикальных и горизонтальных отступов
\titlespacing*{\chapter}{0pt}{-30pt}{*2}
\titlespacing*{\section}{\parindent}{*2}{*1}
\titlespacing*{\subsection}{\parindent}{*2}{*1}
\titlespacing*{\subsubsection}{\parindent}{*0.5}{*0}

\usepackage{polyglossia}
\setdefaultlanguage[spelling=modern]{russian}
\setotherlanguage{english}
\defaultfontfeatures{Mapping=tex-text}
\newfontfamily{\cyrillicfont}{Times New Roman}
\newfontfamily{\cyrillicfonttt}{Courier New} % шрифт URL-ссылок
%\newfontfamily{\sourcecodefont}{Courier New}
\defaultfontfeatures{Ligatures={TeX},Renderer=Basic}    %% свойства шрифтов по умолчанию
\setmainfont[Ligatures={TeX,Historic}]{Times New Roman} %% задаёт основной шрифт документа
%% \setsansfont{CMU Sans Serif}                         %% задаёт шрифт без засечек
\setmonofont{Courier New}                               %% задаёт моноширинный шрифт

\newcommand{\fixedspaceword}[2][1]{%
  \begingroup
  \spaceskip=#1\fontdimen2\font
  \xspaceskip=0pt\relax % just to be sure
  #2%
  \endgroup
}

% объявляем новую команду для переноса строки внутри ячейки таблицы
\newcommand{\spcell}[2][c]{%
  \begin{tabular}[#1]{@{}c@{}}#2\end{tabular}}

\bibliographystyle{gost705}

\makeatletter
\def\@biblabel#1{#1. }
\makeatother

\usepackage{caption}
\AtBeginDocument{%
  \def\contentsname{ОГЛАВЛЕНИЕ}
  \def\bibname{СПИСОК ИСПОЛЬЗОВАННОЙ ЛИТЕРАТУРЫ}
  \renewcommand{\figurename}{Рисунок}
  \renewcommand{\tablename}{Таблица}
  \renewcommand{\thefigure}{\arabic{chapter}.\arabic{figure}}
  \renewcommand{\thetable}{\arabic{chapter}.\arabic{table}}
  \renewcommand{\theequation}{\arabic{chapter}.\arabic{equation}}
}

\author{Великий Студент Анонимович}
\date{\today}

\title{Дипломная работа\\
  <<Методы написания дипломной работы в последний момент перед предзащитой>>}

\hypersetup{
  pdfauthor={Великий С.А.},
  pdftitle={Методы написания дипломной работы в последний момент перед предзащитой},
  pdfkeywords={СПИСОК КЛЮЧЕВЫХ СЛОВ, КАСАЮЩИХСЯ ТЕМЫ РАБОТЫ},
  pdfsubject={Тема работы},
  pdfcreator={xelatex},
  pdflang={Russian}}
