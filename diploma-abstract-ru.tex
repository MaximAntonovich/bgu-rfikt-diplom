Дипломная работа: \total{page}\ страниц, \totalfigures{}~иллюстраций, \totaltables{}\ таблиц,
18 источников, 1 приложение.

\vspace{\baselineskip}

ЭКЗОН, ОБРАБОТКА ДАННЫХ, ИЕРАРХИЧЕСКАЯ КЛАСТЕРИЗАЦИЯ, СПЕКТРАЛЬНАЯ КЛАСТЕРИЗАЦИЯ, РАНЖИРОВАНИЕ ВАЖНОСТИ ПРИЗНАКОВ, СНИЖЕНИЕ РАЗМЕРНОСТИ, ОЦЕНКА РЕЗУЛЬТАТА КЛАСТЕРИЗАЦИИ

\vspace{\baselineskip}

Объектом исследования является разработка алгоритма автоматического ранжирования и отбора наиболее важных признаков экзона онкогена.

Целью работы является разработка и исследование алгоритмов автоматического выбора наиболее значимых признаков экзонов и алгоритмов кластерного анализа экзонов онкогена.

Исследованы подходы определения важности признаков в задаче кластеризации, исследованы и отобраны алгоритмы кластеризации исходных данных, разработана имитационная модель генерации данных, на основе изученных подходов разработан алгоритм определения важности признаков,исследовано влияние предобработки данных на ранжирование признаков, разработан способ оценки и проверки показателя важности.

В результате работы разработан алгоритм отбора наиболее важных признаков экзона онкогена, включающий в себя предобработку данных, процедуру ранжирования и отбора признаков, построение интерпретируемого результата отбора. Проведено обсуждение и анализ полученных данных, произведена оценка работы алгоритма.
