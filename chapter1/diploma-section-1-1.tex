\subsection{Язык программирования Python}

Для реализации процесса отбора признаков был выбран язык Python версии 3.5. Python представляет собой открытый и свободный, широко используемый в индустрии и науке язык программирования. Язык динамически интерпретируемый с сильной динамической типизацией, поддерживает различные парадигмы программирования, такие как функциональное или объектно-ориентированное. Целью, декларированной при создании языка, является обеспечение читаемости, поддерживаемости, уменьшение количества кода при сохранении наглядности. Язык легко изучить, при возникновении трудностей на помощь приходит большое сообщество. Сочетание перечисленных факторов привело к выбору Python в качестве основного и единственного языка реализации проектируемого алгоритма.

\subsection{Экосистема анализа данных Python}

Помимо прикладного программирования, Python распространен в анализе данных. Этому способствует развитая инфраструктура в виде большого количества специализированных библиотек: поддержка линейной алгебры(numpy), статистики, оптимизации и обработки сигналов(scipy), табличных структур данных(pandas), алгоритмов машинного обучения и интеллектуального анализа данных(scikit-learn), визуализации и графиков(matplotlib, seaborn).  Перечисленные библиотеки позволяют организовать процесс анализа данных c меньшими затратами времени и усилий.  

\subsection{Основные результаты и выводы}
Для реализации алгоритма отбора признаков был выбран язык программирования Python, чьи особенности позволяют эффективно решать задачи интеллектуального анализа данных.