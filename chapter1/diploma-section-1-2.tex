\subsection{Модель генерации кластеров}
На стадии разработки алгоритма для проверки способности выбранного подхода выделять кластеры и отсекать неинформативные признаки была разработана модель генерации многомерных объектов по многомерному нормальному распределению. Такой подход при генерации позволяет задать форму кластера как многомерный эллипсоид, а возможность задавать корреляции между признаками позволяет заложить в модель число заранее известных нерепрезентативных признаков, на которых можно проверить работу алгоритмов обзора.

Для оценки работы алгоритмов был сгенерирован набор многомерных данных, включающий 100 объектов с 1500 признаками, из которых 100 признаков генерировались независимо, то есть считались важными в рамках данной модели, а 1400 генерировались как комбинации окажных признаков. Так же был на основе сгенерированного набора данных был построен стандартизированный набор данных, установивший матожидание каждого признака равным 0, а дисперсию равной 1.

\subsection{Разработка алгоритмов кластеризации}
Для реализации алгоритмов кластеризации были отобраны алгоритмы аггломеративной иерахической и спектральной кластеризации. Для иерархической кластеризацираци выбрано расстояние Уорда,лучше остальных альтернатив разделяющее кластеры, как критерий для объединения кластеров, и евклидова метрика как расстояние между отдельными объектами. В спектральной кластеризации применялась нормировка лапласиана средним арифметическим.

\subsection{Разработка алгоритмов отбора признаков}
Как было отмечено ранее, алгоритмы отбора признаков состоят из базовой части вычисления спектра лапласиана графа ближайших соседей и дополнительных действий, разных для каждого алгоритма. Ниже мы рассмотрим действие каждого из этих факторов. 
\subsubsection{Применение ядер к функции ранжирования}
\subsubsection{Отдельный отбор признаков для каждого кластера}
\subsubsection{Неотрицательная факторизация матрицы признаков}
\subsubsection{Регуляризация}
\subsection{Основные результаты и выводы}
Был разработан окончательный вариант кластеризации экзонов и отбора и ранжирования признаков. 