\subsection{Описание CRISP-DM}
CRISP-DM(CRoss-Industry Standart Process for Data Mining) -- независимый от приложений универсальный процессный стадарт для обработки данных. Как любой процессный стандарт, он имеет иерархическую структуру: фазы, общие задачи, специализированные задачи, отдельные процессы. Целью методологии является представление решения задачи как последовательность дискретных событий, допускающий параллельную обработку, плагирование, делегирование. \cite{azevedo2008kdd}

На верхнем уровне процесс обработки данных организован как набор фаз, сменяющих друг друга, каждая фаза состоит из нескольких общих задач. Задачи называются общими, потому что они формулируются с целью покрытия все возможных ситуаций в анализе данных. Решения общих задач должны быть завершенными и стабильными. Завершенными с точки зрения предоставления результата, удовлетворяющего исходной формулировке и стабильными с точки зрения масштабируемости под следующую фазу и совместимости с её задачами.

Уровень специализированных задач конкретизирует общие задачи применительно к конкретному набору данных. Например общей задаче <<подготовка данных>> могут соответствовать специализированные задачи <<стандартизация>>, <<удаление выбросов>>, <<кодирование категориальных переменных>>.

Четвертый уровень содержит процессы решения каждой конкретной задачи как последовательность решений, действий и результатов. Процесс представляет собой развитие действий в каждой задаче.

\subsection{Контекст анализа данных}
CRISP-DM вводит соответствие общими и специализированными уровнями в анализе данных. Рассматривается четыре аспекта анализа данных:
\begin{itemize}
	\item \textbf{Область приложения}, определяет область знания, откуда идет задача
	\item \textbf{Тип задачи анализа данных}, определяет специфическую цел анализа данных, преследуемую проектом
	\item \textbf{Аспект специфики}, определяет возможные сложные ситуации, возникающие в анализе данных
	\item \textbf{Технический аспект}, определяет техники и инструменты анализа данных, применяемые в данной задаче 
\end{itemize}

Каждый контекст включает в себя один или больше перечисленных аспектов. 

\subsection{Модель CRISP-DM}
Модель CRISP-DM представляет собой жизненный цикл проекта анализа данных и содержит все его фазы, задачи, и отношения между фазами и задачами. Отношения могут существовать между данными, между задачами, могут приходить из области приложения и других источников. \cite{azevedo2008kdd}

Модель состоит из шести фаз. Переходы между фазами происходят не только вперёд, переходы назад или переходы через фазу тоже иногда небходимы. Результат каждой фазы определяет следующую фазу или конкретную задачу. Стрелки обозначают наиболее общие и частые переходы между фазами. Рассмотрим фазы CRISP-DM:
\begin{enumerate}
	\item \textit{Понимание цели приложения}

	Начальная фаза предполагает изучения целей проекта в прилагаемой области и формулирование из предметной задачи задачи анализа данных и разработку плана для её решения.
	\item \textit{Понимание данных}

	Вторая фаза начинается со сбора данных, и включает в себя знакомство с содержанием и качеством данных, детектирование аномалий, построение выборок и построение гипотез.
	\item \textit{Подготовка данных}

	Фаза подготовки данных включает в себя все действия, необходимые для подготовки финального исследуемого набора данных из полученных на второй фазе сырых данных. Задания включают в себя миграцию, очистку, выбор атрибутов, трансформацию данных с использованием специализированных инструментов.
	\item \textit{Моделирование}

	На данной фазе иденцифицируются и решаются задачи анализа данных, подбираются целевые величины для оптимизации, проверяются сделанные гипотезы. Часто требуется возвращение ко второй фазе.
	\item \textit{Валидация}

	Фаза состоит в оценке качества построенной модели с точки зрения метрик анализа данных, производится проверка шагов, сделанных при построении модели, связь метрики анализа данных с целями, поставленными в первой фазе. К концу фазы предполагается использование в области приложения результатов модели.
	\item \textit{Развёртывание}

	Создание модели не означает собой конец проекта. Как только модель способна производить знания, её использование должно быть организовано и задокументировано, для удобства конечного пользователя. Так же модель должна быть гибкой и поддерживаемой. Результатом фазы является готовый программный продукт, пригодный к внедрению.
\end{enumerate}  
\subsection{Основные результаты и выводы}
В качестве модели разработки алгоритма отбра признаков была выбрана CRISP-DM, поскольку она обладает следующими преимуществами:
\begin{itemize}
	\item Покрытие полного жизненного цикла решения
	\item Независимость от области знаний источника задачи
	\item Независимость от инструментов анализа данных
	\item Гибкость и нелинейность
	\item Иерархическая структура, декомпозиция задач
	\item Упор на интеллектуальный анализ данных
\end{itemize}