\begin{itemize}
	\item Были рассмотрены актуальные подходы в выделении важных признаков в задаче кластеризации. В качестве основного выбран подход, основанный на спектральной теории графов.
	\item Выбрана модель CRISP-DM для решения поставленной задачи отбора экзонных признаков. Модель характиризуется широкой применимостью и иерархической структурой, позволяющей делить задачи.
	\item Построена модель для генерации кластеризированных данных для разработки алгоритма. Модель генерирует многомерные нормальные кластеры с возможностью задания связей между компонентами.
	\item Были описаны и проанализированы алгоритмы кластеризации и визуализации многомерных данных. В качестве кластеризации ыбл выбран алгоритмы иерархической и спектральной кластеризации, хорошо подходящие для поиска структуры в данных, а в качестве визуализации был выбран алгоритм t-SNE, пригодный для построения низкоразмерного отображения исходных данных.
	\item Построен алгоритм для оценивания результатов кластеризации с помощью теоретико-информационных подходов
	\item Построен алгоритм отбора признаков, включающий различные подходы и оценку результата, в том числе комбинаторную оптимизацию
	\item Обоснован выбор технических средств для реализации алгоритма
	\item В экспериментальных данных выделены наиболее важные признаки, задающие кластерную структуру.
	\item Установлен факт ненормальности экзонных признаков и влияние факта на процесс отбора признаков
	\item Подготовлены направления для дальнейшего развития исследования
	\item Работа выполнена при участии в НИР 834/18
	\item Результаты исследования по кластеризации экзонов онкогена доложены на международной конференции CSIST'2016, статья опубликована в сборнике трудов конференции
\end{itemize}