С середины XX века развитие методов информатики достигло этапа, позволяющего их внедрение в другие области человеческого знания. Методы
информатики начали активно применяться в биологии: увеличение производительности компьютеров вкупе с ростом количества экспериментальных
данных привело к образованию и бурному развитию науки биоинформатики\cite{durbin1998biological}.

В частности, одной из задач биоинформатики является анализ символьных последовательностей биологического происхождения, к которым относятся геномные последовательности\cite{oyelade2016clustering}.

ДНК(дезоксирибонуклеиновая кислота) – макромолекула, обеспечивающая хранение и передачу из поколения в поколение наследственной информации развития и функционирования живых организмов. ДНК содержит информацию о структуре различных видов РНК(рибонуклеиновых кислот) и белков, закодированую последовательностью из повторяющихся блоков – нуклеотидов\cite{sakharkar2004distributions}.

Ген -- участок молекулы ДНК. Согласно современным представлениям, каждый ген отвечает за синтез функциональной молекулы белка или РНК. При этом каждому гену может соответствовать несколько участков ДНК, необязательно расположенных последовательно. Кодирующие участки могут кодироваться с некодирующими. Кодирующие участки, копии которых составляют зрелую РНК называются экзонами, а некодирующие, то есть игнорируемые, интронами. В результате процесса, называемого сплайсингом, интроны вырезаются из пре-РНК\cite{grinev2015decoding}.

У высших организмов транскрипт первичной РНК может подвергаться альтернативному сплайсингу. В этом случае некоторые экзоны могут удаляться вместе с интронами. У каждого организма удаляемые экзоны разные, соответственнно могут образовываться различные варианты белков, соответствующие одному гену. Количество обнаруженных вариантов белка может достигать десятков тысяч, но как правило количество теоретически возможных вариантов, посчитанное по комбинаторным принципам, бывает ещё больше и достигает сотен тысяч. 

Анализ возможных правил альтернативного сплайсинга и предсказывание получаемого белка становятся важным инструментом в профилактике онкозаболеваний. Однако данный анализ затруднен необходимостью обрабатывать сравнительно большие объемы данных, соответствующие характеристикам экзонов и образуемых ими последовательностей. Данная работа исследует применимость направления интеллектуального анализа данных, называемого отбором признаков, для уменьшения количества исследуемых признаков экзона при сохранении репрезентативности. Отбор признаков характеризуется ранжированием признаков по важности и отбору наиболее важных, то есть несущих в себе информацию о всем экзоне.

Целью данной работы является разработка и исследование алгоритмов автоматического выбора наиболее значимых признаков экзонов и алгоритмов кластерного анализа экзонов онкогена. Цели соответствуют следующие задачи:
\begin{enumerate}
	\item Литературный обзор
	\item Разработка и реализация алгоритмов автоматического выбора атрибутов объектов
	\item Разработка и реализация алгоритмов кластерного анализа
	\item Разработка и реализация простейшей имитационной модели генерации кластеров многомерных данных
	\item Сравнительный анализ разработанных алгоритмов на смоделированных и экспериментальных данных
\end{enumerate}

Работа делится на три главы. В главе 1 проводится обзор подходов к кластеризации и отбору признаков при кластеризации и выбираются алгоритмы для реализации. В главе 2 производится разработка алгоритмов и выполнение задач 2, 3, 4. И глава 3 предлагает сравнение разработанного алгоритма на реальных и смоделированных экзонах. В конце делается заключение о применимости предлагаемого подхода к данным экзонам.