Дыпломная праца: \total{page}\ старонак, \totalfigures{}~iллюстрацый, \totaltables{}~таблiц,
18 крынiц, 1 дадатак.

% ў Ў

\vspace{\baselineskip} 
ЭКЗОН, АПРАЦОЎКА ДАДЗЕНЫХ, ІЕРАРХІЧНАЯ КЛАСТЭРЫЗАЦЫЯ, СПЕКТРАЛЬНАЯ КЛАСТЭРЫЗАЦЫЯ, РАНЖЫРАВАННЕ ВАЖНАСЦІ АЗНАК, ЗНІЖЭННЕ ПАМЕРНАСЦІ, АДЗНАКА РЭЗУЛЬТАТУ КЛАСТЭРЫЗАЦЫІ

\vspace{\baselineskip}

Аб'ектам даследавання з'яўляецца распрацоўка алгарытму аўтаматычнага ранжыравання і адбору найбольш важных азнак экзонаў анкагена.

Мэтай працы з'яўляецца распрацоўка і даследаванне алгарытмаў аўтаматычнага выбару найбольш значных азнак экзонаў і алгарытмаў кластэрнага аналізу экзонаў анкагена.

Даследаваны падыходы вызначэння важнасці азнак ў задачы кластарызацыі, даследаваны і адабраны алгарытмы кластарызацыі зыходных дадзеных, распрацавана імітацыйная мадэль генерацыі дадзеных, на аснове вывучаных падыходаў распрацаваны алгарытм вызначэння важнасці азнак, даследавана ўплыў прадапрацоўкі дадзеных на ранжыраванне прыкмет, распрацаваны спосаб ацэнкі і праверкі паказчыка важнасці.

У выніку працы распрацаваны алгарытм адбору найбольш важных прыкмет азнак экзонаў анкагена, які ўключае ў сябе прадапрацоўку дадзеных, працэдуру ранжыравання і адбору азнак, пабудова інтэрпрэтаванага выніку адбору. Праведзена абмеркаванне і аналіз атрыманых дадзеных, праведзеная ацэнка работы алгарытму.
